\documentclass[a4paper, 12pt]{extreport}
\usepackage[margin=1in]{geometry}
\usepackage{tikz}
\usepackage{setspace}
\graphicspath{ {../../resources/images/} }
\usepackage[titles]{tocloft}
\usepackage{titlesec}
\usepackage[hidelinks]{hyperref}
\usepackage{pdfpages}
\titleformat{\chapter}[hang]{\huge\bfseries}{\thechapter}{20pt}{\vspace{0.5em}}
\titlespacing*{\chapter}{0pt}{-3em}{1.1\parskip}
\usepackage[backend=biber, style=ieee]{biblatex}
\usepackage{parskip}
\setlength{\parindent}{1cm}
\setlength{\parskip}{.5\baselineskip}
\usepackage[nottoc,numbib]{tocbibind}
\addbibresource{../../resources/citation.bib}

\begin{document}
	
	\onehalfspacing
	
	\begin{titlepage}
		
		\begin{tikzpicture}[remember picture, overlay]
			\node[xshift=14cm,yshift=-1.8cm,anchor=north west] at (current page.north west){%
				\includegraphics[height=2.5cm]{logos/sunway}};
		\end{tikzpicture}
		
		\vfill
		
		\begin{center}
			\textbf{\large CAPSTONE PROJECT 1} \\
			\textbf{\large Planning Document} \\
			\vspace{1cm}
			\textbf{\large Evaluation of Nature-inspired Optimisation\\Algorithms in Solving Versus Tetris}
			
			\vspace{1cm}
			
			by
			
			\vspace{1cm}
			
			\large Yap Wei Xiang \\
			21067939
			
			\vspace{1cm}
			
			\large Supervisor: Dr Richard Wong Teck Ken
			
			\vspace{1cm}
			
			\normalsize Semester: April 2024 \\
			Date: % DATE OF SUBMISSION
			
			\vfill
			
			Department of Computing and Information Systems\\
			School of Engineering and Technology\\
			Sunway University
		\end{center}
		
	\end{titlepage}
	
	\pagenumbering{roman} % switch to roman numerals
	
	\chapter*{Abstract}
	
	\addcontentsline{toc}{chapter}{Abstract}
	
	\tableofcontents
	
	\chapter{Introduction}
	
		\pagenumbering{arabic}
		
		% What is Tetris		
		Tetris is a popular video game created in 1984 by computer programmer Alexey Pajitnov  \cite{about-tetris}. It is a puzzle game that requires players to strategically place sequences of pieces known as "Tetriminos" into a rectangular Matrix. In the classic game, players attempt to clear as many lines as possible by completing horizontal rows of blocks without empty space, but if the Tetriminos surpass the top of the Matrix, the game is over.
		
		% Tetris and research
		
		% Tetris and computer science
		
		% Versus Tetris
		
		\section{Motivation}
		
			% In this section, you'll explain why your capstone project is important and relevant. What inspired you to pursue this topic? Is there a particular problem or challenge in the field that you're interested in addressing?
			% Consider discussing the broader significance of your research topic, such as its potential impact on society, industry, or academia. Why should readers care about your project?
			% You might also highlight any personal or professional experiences that motivated your interest in the topic.
		
		\section{Problem Statement}
			
			% Here, you'll clearly define the problem or issue that your capstone project aims to address. What specific challenge or question are you seeking to answer?
			% Describe the current state of the problem and any existing solutions or approaches. What limitations or shortcomings do these solutions have?
			% Be concise and specific in articulating the problem statement, making it clear to readers what your project seeks to contribute to the field.
		
		\section{Aim}
		
		\section{Objectives}
		
		\section{Project Scope}
		
		% Exceptional overview of the proposed project.
		% The problem statement, objectives, scope of work, methodology, proposed outcome and timeline are written in a clear precise manner and presented in proper order.
	
	\chapter{Literature Review}
		
		% A well-articulated introduction that provides a clear, logical, and succinct description of content, scope, and organization of the review which draws the reader's attention to a central concern, debate or contention.
		% Body of section includes citations to a range of reliable sources that critically substantiate and contextualizes all major claims made.
		% Exceptional discussion that summarizes the body of review, highlights the most important findings (in your opinion) and its implication to the direction of the project.
		% Section is very well organized.
		% The review has clarity, simplicity, parsimony, which includes clear transitions and systematic use of headings and subheadings.
	
	\chapter{Technical Plan}
	
		% Graphical and textual description is provided for flow of information between components in the system/stages in the research study.
		% Graphical overview of system/research study corresponds to the textual description provided with no errors.	
		%Excellent description is provided for all methodologies, tools and techniques used to specify, design, build, test, integrate, document and deliver work products throughout the project along with the relevant supporting documents.
		% All documents provided are well-drawn/well-written, complete and correct. The purpose and role of each document in the project is stated clearly.
	
	\chapter{Work Plan}
		
		% Excellent description of all work activities to be performed during the project, relationship among the activities and the resultant work products for each activity.
		% Scheduled duration for each work activity is justified and supported by the identified project risk factors and work decomposition that each activity requires.
		% The whole project schedule, milestones and activitiy lists are depicted professionally on the Gantt chart with all dependencies, predecessor and successor work activities denoted clearly.
	
	\printbibliography[heading={bibnumbered}, title={References}]
		
\end{document}