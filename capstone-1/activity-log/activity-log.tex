\documentclass[a4paper, 12pt]{extreport}
\usepackage[margin=1in]{geometry}
\usepackage{parskip}
\setlength{\parindent}{1cm}
\setlength{\parskip}{.5\baselineskip}
\usepackage{tikz}
\usepackage{setspace}
\graphicspath{ {../../resources/images/} }
\usepackage[titles]{tocloft}
\usepackage{titlesec}
\usepackage[hidelinks]{hyperref}
\usepackage{pdfpages}
\titleformat{\chapter}[hang]{\huge\bfseries}{\thechapter}{20pt}{\vspace{0.5em}}
\titlespacing*{\chapter}{0pt}{-3em}{1.1\parskip}
\usepackage[backend=biber, style=ieee]{biblatex}
\usepackage[nottoc,numbib]{tocbibind}
\usepackage{pgfgantt}
\usepackage{pdflscape}
\usepackage{tabularray,varwidth,enumitem}
\renewbibmacro{finentry}{\finentry%
	\iffieldundef{annotation}
	{}
	{\par\medskip\printfield{annotation}\medskip\finentry}}

\addbibresource{../../resources/citation.bib}

\begin{document}
	
	\onehalfspacing
	
	\begin{titlepage}
	
	\centering
	\begin{tikzpicture}
		\node[] at (current page.north west){%
			\includegraphics[height=3.5cm]{logos/sunway-set}};
	\end{tikzpicture}
	
	\vspace{.5cm}
	
	\begin{center}
		\textbf{\large CAPSTONE PROJECT 1} \\
		\textbf{\large Activity Log} \\
		\vspace{1cm}
		\textbf{\large Playing Versus Tetris with \\ Nature-inspired Optimisation Algorithms}
		
		\vspace{1cm}
		
		by
		
		\vspace{1cm}
		
		\large Yap Wei Xiang \\
		21067939
		
		\vspace{1cm}
		
		Bachelor of Science (Honours) in Computer Science
		
		\vspace{1cm}
		
		\large Supervisor: Dr Richard Wong Teck Ken
		
		\vspace{1cm}
		
		\normalsize Semester: April 2024 \\
		Date: 1 August 2024
		
		\vfill
		
		Department of Computing and Information Systems\\
		School of Engineering and Technology\\
		Sunway University
	\end{center}
	
\end{titlepage}
	
	\pagenumbering{roman}
	\tableofcontents
	
	\chapter{Timeline}
	
		\pagenumbering{arabic}
		
		% Project is split into critical stages. Brief description is provided for each stage of the project on the processes to be executed.
		% Time allocated for each stage is justified by student assessment of own workload, ability and work to be done to deliver.
		
		In this chapter, the project's progression is meticulously documented. In these pages, the critical stages of the project are depicted in the form of tables. A Gantt chart of the timeline is also provided to give an overview of the project's temporal progression.
		
		\begin{longtblr}[
			caption = {Overall Work Activities},
			label = {tab:work},
			]{| c | X | X | X | c |}
			\hline
			Phases & Work Activities & Work Product & Risk Factors & {Time \\ Allocated} \\
			\hline
			Introduction & Write a comprehensive introduction, including the motivation, problem statement, aim, objectives and scope of the project. & A comprehensive introduction & Misunderstanding of concepts & 1 Week \\
			\hline
			{Literature \\ Review} & Find and read a wide variety of literature to write a comprehensive review that (1) justifies the use of non-traditional algorithms and (2) showcase past approaches to the game. & A comprehensive literature review that has a diverse set of sources & Lack of understanding of concepts, misunderstanding concepts & 12 Weeks \\
			\hline
			{Methodology} & Come up with a methodology that covers all bases, it should include rule definition, algorithm selection, problem formulation and evaluation metrics & A well-defined methodology & Inadequate explanation of rules, trouble formulating problem formally & 4 Weeks \\
			\hline
			{Work Plan} & Create a work plan that includes this table, a Gantt chart summarising what has been done in the planning phase of the project, as well as a similar table and Gantt chart for the implementation phase. & A well-defined work plan with goals to be met & Time constraints, formatting issues & 1 Week \\
			\hline
		\end{longtblr}
		
		\begin{landscape}
			\begin{figure}
				\centering
				\begin{ganttchart}[x unit=1cm, vgrid=dotted, hgrid = true]{1}{15}
					\gantttitle{W1}{1}
					\gantttitle{W2}{1}
					\gantttitle{W3}{1}
					\gantttitle{W4}{1}
					\gantttitle{W5}{1}
					\gantttitle{W6}{1}
					\gantttitle{W7}{1} 
					\gantttitle{W8}{1}
					\gantttitle{W9}{1}
					\gantttitle{W10}{1}
					\gantttitle{W11}{1}
					\gantttitle{W12}{1}
					\gantttitle{W13}{1}
					\gantttitle{W14}{1}
					\gantttitle{W15}{1}\\
					\ganttgroup{Introduction}{1}{1}\\
					\ganttgroup{Literature Review}{1}{12}\\
					\ganttbar{Research}{1}{10}\\
					\ganttlinkedbar{Writing}{3}{12}\\
					\ganttmilestone{Completed Review}{12}\\
					\ganttgroup{Methodology}{12}{15}\\
					\ganttbar{Defining Rules}{12}{14}\\
					\ganttbar{On Implementation}{14}{15}\\
					\ganttgroup{Work Plan}{15}{15}\\
				\end{ganttchart}
				\caption{CP1 Timeline}
			\end{figure}
		\end{landscape}
		
		\section{Weekly Breakdown}
		
			In this section, a week-to-week overview that summarises work done that particular week will be shown in table form. Each subsection will contain a table that showcases work done for a particular chapter of the planning document. 
		
		\subsection{Introduction}
		\label{sec:intro}
		
			The introduction serves as the project's foundation, providing essential background information, introducing the topic, and articulating the project's objectives. Recognising its pivotal role, \textbf{one week} of time was allocated for its composition. This deliberate time-frame aimed to allow ample time for thoroughness, ensuring no essential elements were overlooked in writing a comprehensive and compelling introduction.
			
			\begin{longtblr}[
				caption = Weekly breakdown of work done for Introduction Chapter.
				]{colspec={c|X}, hlines, vlines}
				Week & Work Done \\
				1 & \begin{varwidth}[t]{\linewidth}
					\begin{itemize}[topsep=0pt,parsep=0pt]
						\item Researched existing literature on Tetris, NP-completeness and Nature-inspired algorithms.
						\item Wrote the introduction chapter.
						\item Learnt the tikz \LaTeX package to draw better figures.
					\end{itemize}
				\end{varwidth}\\
			\end{longtblr}

		\subsection{Literature Review}
		\label{sec:litrev}
		
			The literature review is one of the chapters that will take up a lot of space in the planning document. It is important to be thorough and correct about the information written on these pages. As such, a generous time of \textbf{twelve weeks} were allocated to research and write this section.
			
			\begin{longtblr}[
				caption = Weekly breakdown of work done for Literature Review Chapter.
				]{colspec={c|X}, hlines, vlines}
				Week & Work Done \\
				1 & \begin{varwidth}[t]{\linewidth}
					\begin{itemize}[topsep=0pt,parsep=0pt]
						\item Found and read literature on NP-completeness.
						\item Tried to understand computational complexity as a whole.
					\end{itemize}
				\end{varwidth}\\
				2 & \begin{varwidth}[t]{\linewidth}
					\begin{itemize}[topsep=0pt,parsep=0pt]
						\item Continued researching NP-completeness.
						\item Understood enough computational complexity for the project: complexity classes, reductions, P and NP, etc.
					\end{itemize}
				\end{varwidth}\\
				3 & \begin{varwidth}[t]{\linewidth}
					\begin{itemize}[topsep=0pt,parsep=0pt]
						\item Found and read literature on the difficulty of Tetris.
						\item Began writing the first section of the literature review, dedicated to showing the intrinsic difficulty of the game of Tetris. 
					\end{itemize}
				\end{varwidth}\\
				4 & \begin{varwidth}[t]{\linewidth}
					\begin{itemize}[topsep=0pt,parsep=0pt]
						\item Finished writing first section of literature review.
						\item Created images for clarity.
					\end{itemize}
				\end{varwidth}\\
				5 & \begin{varwidth}[t]{\linewidth}
					\begin{itemize}[topsep=0pt,parsep=0pt]
						\item Found literature that utilise different approaches to playing Tetris with AI.
					\end{itemize}
				\end{varwidth}\\
				6 & \begin{varwidth}[t]{\linewidth}
					\begin{itemize}[topsep=0pt,parsep=0pt]
						\item Continued finding different approaches taken to play Tetris.
						\item Specifically read and digested paper on imitation learning.
						\item Began structuring second section of the literature review, dedicated to showcasing some of the approaches taken to play Tetris that aren't nature-inspired.
					\end{itemize}
				\end{varwidth}\\
				7 & \begin{varwidth}[t]{\linewidth}
					\begin{itemize}[topsep=0pt,parsep=0pt]
						\item Completed a subsection on the use of imitation learning for playing Tetris
						\item Read up on Monte-Carlo Tree Search, UCB1, and how it is used as an approach to Tetris.
					\end{itemize}
				\end{varwidth}\\
				8 & \begin{varwidth}[t]{\linewidth}
					\begin{itemize}[topsep=0pt,parsep=0pt]
						\item Completed a subsection on the use of UCB1 for playing Tetris.
						\item Read up on some deep reinforcement learning concepts because some papers were found that used Q-learning to play Tetris.
					\end{itemize}
				\end{varwidth}\\
				9 & \begin{varwidth}[t]{\linewidth}
					\begin{itemize}[topsep=0pt,parsep=0pt]
						\item Read literature that described the use of deep reinforcement learning for Tetris.
						\item Completed a subsection on the use of deep reinforcement learning for playing Tetris.
						\item Found and read papers that utilise non nature-inspired meta-heuristics like Harmony Search Algorithm and MVP Algorithm to play Tetris.
						\item Found literature on using Genetic Algorithms and Particle Swarm Optimisation to play Tetris.
					\end{itemize}
				\end{varwidth}\\
				10 & \begin{varwidth}[t]{\linewidth}
					\begin{itemize}[topsep=0pt,parsep=0pt]
						\item Found and read literature on the use of Ant Colony Optimisation to play Tetris.
						\item Completed a subsection on the use of different meta-heuristic algorithms for playing Tetris.
						\item Started writing section dedicated to demonstrating the use of nature-inspired algorithms in playing Tetris.
					\end{itemize}
				\end{varwidth}\\
				11 & \begin{varwidth}[t]{\linewidth}
					\begin{itemize}[topsep=0pt,parsep=0pt]
						\item Completed a subsection on the use of genetic algorithms in playing Tetris.
					\end{itemize}
				\end{varwidth}\\
				12 & \begin{varwidth}[t]{\linewidth}
					\begin{itemize}[topsep=0pt,parsep=0pt]
						\item Completed a subsection on the use of Particle Swarm Optimisation in playing Tetris.
						\item Completed a subsection on the use of Ant Colony Optimisation in playing Tetris.
					\end{itemize}
				\end{varwidth}\\
			\end{longtblr}
		
		\subsection{Methodology}
		
			The methodology is arguably the most important chapter of the review as it sets up the entire project in the implementation phase (CP2). \textbf{Four weeks} were allocated to write up the methodology, which should define everything that will be done in a structured manner.
			
			\begin{longtblr}[
				caption = Weekly breakdown of work done for Methodology Chapter.
				]{colspec={c|X}, hlines, vlines}
				Week & Work Done \\
				12 & \begin{varwidth}[t]{\linewidth}
					\begin{itemize}[topsep=0pt,parsep=0pt]
						\item Structured the Methodology chapter
						\item Began writing section on rules, defining the piece randomiser and matrix dimensions.
					\end{itemize}
				\end{varwidth}\\
				13 & \begin{varwidth}[t]{\linewidth}
					\begin{itemize}[topsep=0pt,parsep=0pt]
						\item Researched Tetris spin rules.
						\item Continued writing rules section, defining spins, gravity and other features.
					\end{itemize}
				\end{varwidth}\\
				14 & \begin{varwidth}[t]{\linewidth}
					\begin{itemize}[topsep=0pt,parsep=0pt]
						\item Researched Tetris attack rules.
						\item Completed writing rules section, adding attack rules to the rules.
						\item Completed section on algorithm selection.
						\item Began problem formulation section.
					\end{itemize}
				\end{varwidth}\\
				15 & \begin{varwidth}[t]{\linewidth}
					\begin{itemize}[topsep=0pt,parsep=0pt]
						\item Completed problem formulation section.
						\item Completed section on evaluation metrics.
					\end{itemize}
				\end{varwidth}\\
			\end{longtblr}
		
		\subsection{Work Plan}
			
			Careful consideration needs to be taken when coming up with a work plan. Risk factors need to be thought of and time needs to be allocated to each task. However, since the work plan takes up the least space in the document, only \textbf{one week} of time was allocated to it.
			
			\begin{longtblr}[
				caption = Weekly breakdown of work done for Work Plan  Chapter.
				]{colspec={c|X}, hlines, vlines}
				Week & Work Done \\
				15 & \begin{varwidth}[t]{\linewidth}
					\begin{itemize}[topsep=0pt,parsep=0pt]
						\item Came up with tasks that need to be done.
						\item Created table to showcase work activities, risk factors, and time allocated for all tasks.
						\item Created Gantt chart to visualise the timeline.
					\end{itemize}
				\end{varwidth}\\
			\end{longtblr}
		
	% Bibliography
	\nocite{*}
	\printbibliography[heading={bibnumbered}, title={Bibliography}]
	
	\chapter{Meeting Records}
		
		% Attended all scheduled meetings with well-prepared presentations or documents to aid discussion of the project.
		% Sets own goals and able to negotiate deliverables for upcoming meeting based on own workload and ability.
		% Delivers all work as promised along with detailed analysis on the cause/effect of own actions to the project outcome. (Able to do and analyse results of own actions)
		Throughout the semester, I had seven meetings with Dr Richard. Each time we discussed the work that had been done between the current and previous meeting. Dr Richard gave me plenty of pointers, as well as personal advice to produce a better planning document.
		
		The following pages contain all meeting records, including the date and time of the meeting, the items discussed, and the expected work outcomes before the next meeting. 
		
		\includepdf[addtotoc={1, section, 1, Meeting 1, meeting1}, pagecommand={\thispagestyle{plain}}]{./meeting-records/signed/meeting-1}
		
		\includepdf[addtotoc={1, section, 1, Meeting 2, meeting2}, pagecommand={\thispagestyle{plain}}]{./meeting-records/signed/meeting-2}
		
		\includepdf[addtotoc={1, section, 1, Meeting 3, meeting3}, pagecommand={\thispagestyle{plain}}]{./meeting-records/signed/meeting-3}
		
		\includepdf[addtotoc={1, section, 1, Meeting 4, meeting4}, pagecommand={\thispagestyle{plain}}]{./meeting-records/signed/meeting-4}
		
		\includepdf[addtotoc={1, section, 1, Meeting 5, meeting5}, pagecommand={\thispagestyle{plain}}]{./meeting-records/signed/meeting-5}
		
		\includepdf[addtotoc={1, section, 1, Meeting 6, meeting6}, pagecommand={\thispagestyle{plain}}]{./meeting-records/signed/meeting-6}
		
		\includepdf[addtotoc={1, section, 1, Meeting 7, meeting7}, pagecommand={\thispagestyle{plain}}]{./meeting-records/signed/meeting-7}
		
\end{document}