\documentclass[a4paper, 12pt]{extreport}
\usepackage[margin=1in]{geometry}
\usepackage{parskip}
\setlength{\parindent}{1cm}
\setlength{\parskip}{.5\baselineskip}
\usepackage{tikz}
\usepackage{setspace}
\graphicspath{ {../../resources/images/} }
\usepackage[titles]{tocloft}
\usepackage{titlesec}
\usepackage[hidelinks]{hyperref}
\usepackage{pdfpages}
\titleformat{\chapter}[hang]{\huge\bfseries}{\thechapter}{20pt}{\vspace{0.5em}}
\titlespacing*{\chapter}{0pt}{-3em}{1.1\parskip}
\usepackage[backend=biber, style=ieee]{biblatex}
\usepackage[nottoc,numbib]{tocbibind}
\usepackage{pgfgantt}
\renewbibmacro{finentry}{\finentry%
	\iffieldundef{annotation}
	{}
	{\par\medskip\printfield{annotation}\medskip\finentry}}

\addbibresource{../../resources/citation.bib}

\begin{document}
	
	\onehalfspacing
	
	\begin{titlepage}
		
		\begin{tikzpicture}[remember picture, overlay]
			\node[xshift=14cm,yshift=-1.8cm,anchor=north west] at (current page.north west){%
			\includegraphics[height=2.5cm]{logos/sunway}};
		\end{tikzpicture}
		
		\vfill
		
		\begin{center}
			\textbf{\large CAPSTONE PROJECT 1} \\
			\textbf{\large Activity Log} \\
			\vspace{1cm}
			\textbf{\large Evaluation of Nature-inspired Optimisation\\Algorithms in Solving Versus Tetris}
			
			\vspace{1cm}
			
			by
			
			\vspace{1cm}
			
			\large Yap Wei Xiang \\
			21067939
			
			\vspace{1cm}
			
			\large Supervisor: Dr Richard Wong Teck Ken
			
			\vspace{1cm}
			
			\normalsize Semester: April 2024 \\
			Date: % DATE OF SUBMISSION
			
			\vfill
			
			Department of Computing and Information Systems\\
			School of Engineering and Technology\\
			Sunway University
		\end{center}
		
	\end{titlepage}
	
	\pagenumbering{roman}
	\tableofcontents
	
	\chapter{Timeline}
	
		\pagenumbering{arabic}
		
		% Project is split into critical stages. Brief description is provided for each stage of the project on the processes to be executed.
		% Time allocated for each stage is justified by student assessment of own workload, ability and work to be done to deliver.
		
		In this chapter, the project's progression is meticulously documented. In these pages, the critical stages of the project are depicted, each accompanied by a concise description illuminating the strategies and rationales behind the allocated times.
		
		Furthermore, the week-by-week tasks undertaken are outlined within these sections. Thus, offering a granular insight into the day-to-day endeavours taken to propel the project towards its culmination.
		
		\begin{figure}[h]
			\noindent
			\begin{ganttchart}[
				expand chart=\textwidth, 
				time slot format = isodate]{2024-04-22}{2024-08-22}
				\gantttitlecalendar{year, month=name}\\
				\ganttgroup{Introduction}{2024-04-22}{2024-05-03}\\
				\ganttbar{Research}{2024-04-22}{2024-04-26}\\
				\ganttbar{Writing}{2024-04-22}{2024-05-03}\\
			\end{ganttchart}
			\caption{Gantt Chart of Timeline}
		\end{figure}
		
		\section{Writing an Introduction}
		
			The introduction serves as the project's foundation, providing essential background information, introducing the topic, and articulating the project's objectives. Recognizing its pivotal role, a generous time span of \textbf{up to two weeks} was allocated for its composition. This deliberate time-frame aimed to allow ample time for thoroughness, ensuring no essential elements were overlooked in writing a comprehensive and compelling introduction.
			
			\subsection{Week 1}
		
		\section{Conducting the Literature Review}
		
		\section{Coming Up With a Technical Plan}
		
		\section{Making Revisions}
		
		
	% Bibliography
	\nocite{*}
	\printbibliography[heading={bibnumbered}, title={Bibliography}]
	
	\chapter{Meeting Records}
		
		% Attended all scheduled meetings with well-prepared presentations or documents to aid discussion of the project.
		% Sets own goals and able to negotiate deliverables for upcoming meeting based on own workload and ability.
		% Delivers all work as promised along with detailed analysis on the cause/effect of own actions to the project outcome. (Able to do and analyse results of own actions)
		
		
		\includepdf[addtotoc={1, section, 1, Template, template1}, pagecommand={\thispagestyle{plain}}]{./meeting-records/template/meeting-record-template}
		
\end{document}